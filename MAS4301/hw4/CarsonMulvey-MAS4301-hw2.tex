\documentclass[11pt,letterpaper]{article}
%\usepackage{fullpage}
\usepackage{amsmath}
\usepackage{amsfonts}
\usepackage{amssymb}
\usepackage{amsthm}
\usepackage{array}
\usepackage[dvipsnames]{xcolor}
\usepackage{textcomp}

\linespread{1.5}

\newcommand{\N}{\mathbb{N}}
\newcommand{\Mod}[1]{\ (\mathrm{mod}\ #1)}

\newenvironment{solution}{\color{blue}\textit{Solution.}}{\color{black}}

\begin{document}
\begin{center}
    \begin{large}
        \textbf{Homework 4} \\
        MAS4301 \\
        Carson Mulvey
    \end{large}
\end{center}

\begin{enumerate}
\item[\textbf{NB 4.1}] We claim that $\phi$ is an isomorphism. It suffices to show that $\phi$ is one-to-one, onto, and operation-preserving.
    
    (One-to-one) Let $\pi_a,\pi_b\in{\text{Perm}(A)}$ satisfy $\phi(\pi_a)=\phi(\pi_b)$. Then
    \begin{align*}
    f^{-1} \circ \pi_a \circ f &= f^{-1} \circ \pi_b \circ f \\
    \implies (f \circ f^{-1}) \circ \pi_a \circ (f \circ f^{-1}) &= (f \circ f^{-1}) \circ \pi_b \circ (f \circ f^{-1}) \\
    \implies \pi_a &= \pi_b,
    \end{align*}
    so $\phi$ is one-to-one.
    
    (Onto) Now let $\pi'\in{S_n}$. We must find a $\pi\in{\text{Perm}(A)$ such that $\phi(\pi)=\pi'$. If such a $\pi$ were to exist, then
    \begin{align*}
        f^{-1} \circ \pi \circ f &= \pi' \\
        \implies (f \circ f^{-1}) \circ \pi \circ (f \circ f^{-1}) &= f \circ \pi' \circ f^{-1} \\
        \implies \pi &= f \circ \pi' \circ f^{-1}.
    \end{align*}
    But we see that since $f$ maps from $\{1,2,\dots,n\}$ to $A$, it follows that  $f \circ \pi' \circ f^{-1} \in \text{Perm}(A)$. Thus, taking $\pi &= f \circ \pi' \circ f^{-1}$ as solved, we see that such a $\pi$ always exists.
    
    (Operation-preserving) Let $\pi_a$ and $\pi_b$ be in ${\text{Perm}(A)}$. Then
    \begin{align*}
    \phi(\pi_a\circ\pi_b) &= f^{-1} \circ \pi_a\circ\pi_b \circ f \\
    &= f^{-1} \circ \pi_a \circ (f \circ f^{-1}) \circ \pi_b \circ f \\
    &= (f^{-1} \circ \pi_a \circ f) \circ (f^{-1} \circ \pi_b \circ f) \\
    &= \phi(\pi_a) \circ \phi(\pi_b),
    \end{align*}
    so $\phi$ is operation-preserving.
    \qed
    
\item[\textbf{6.4}] We have $U(8)=\{1,3,5,7\}$ and $U(10)=\{1,3,7,9\}$. Noting that $U(10)=\{3^0,3^1,3^3,3^2\}=\langle 3 \rangle$, we see that $U(10)$ is cyclic.
However, note that for $U(8)$,
\begin{align*}
    \langle 1 \rangle &= \{1\}, \\
    \langle 3 \rangle &= \{1,3\}, \\
    \langle 5 \rangle &= \{1,5\}, \\
    \langle 7 \rangle &= \{1,7\}.
\end{align*}
Since none of the elements of $U(8)$ are generators, $U(8)$ is not cyclic. By Theorem 6.3, a cyclic and noncyclic group cannot be isomorphic, so $U(10)$ and $U(8)$ cannot be isomorphic.


\item[\textbf{6.12}] Suppose that $\alpha(g)=g^{-1}$ for all $g\in{G}$ is an automorphism. Then for $f,g\in{G}$,
\begin{align*}
    fg &= \alpha((fg)^{-1}) \\
    &= \alpha(g^{-1}f^{-1}) \\
    &= \alpha(g^{-1})\alpha(f^{-1}) \\
    &= gf,
\end{align*}
so $G$ is abelian.

Now suppose that $G$ is abelian. Since $\alpha^2$ is an identity mapping, we have $\alpha^{-1}=\alpha$, so $\alpha$ is a bijection. Also, for $f$ and $g$ in $G$, we have
\begin{align*}
\alpha(fg) &= (fg)^{-1} \\
&= g^{-1}f^{-1} \\
&= f^{-1}g^{-1} \\
&= \alpha(f)\alpha(g).
\end{align*}
We see that $\alpha$ is operation-preserving, so $\alpha$ is an automorphism. \qed

\item[\textbf{6.32}] If $\phi(x)=9x \Mod{50}$, then $\phi$ is an automorphism. Note that $\phi(7)=63 \Mod{50} = 13$ holds.

\item[\textbf{6.34}] We see that $\phi(k_1)$ and $\phi(k_2)$ are in $\phi(K)$ if $k_1,k_2\in{K}$. Then, by operation preservation of an isomorphism, we have $\phi(k_1)\phi(k_2)^{-1}=\phi(k_1k_2^{-1})$. Since $K$ is a subgroup, $K$ is closed under both inverses and the group operation in $G$, so $k_1k_2^{-1}\in{K}$. Thus $\phi(k_1)\phi(k_2)^{-1}$ is in $\bar{G}$. Then since $\phi(k_1)$ and $\phi(k_2)$ being in $\bar{G}$ implies that $\phi(k_1)\phi(k_2)^{-1}$ is in $\bar{G}$, by the One-Step Subgroup Test, $\phi(K)$ is a subgroup of $\bar{G}$. \qed

\item[\textbf{6.42}] Because $\phi^2$ is an identity mapping, we have $\phi=\phi^{-1}$, so $\phi$ is a bijection. Also, for $(a_1,a_2,\dots,a_n)$ and $(b_1,b_2,\dots,b_n)$ in $\mathbb{R}^n$, we have
\begin{align*}
\phi((a_1,\dots,a_n)+(b_1,\dots,b_n)) &= \phi((a_1+b_1,\dots,a_n+b_n) \\
&= (-a_1-b_1,\dots,-a_n-b_n) \\
&= (-a_1,\dots,-a_n) + (-b_1,\dots,-b_n) \\
&= \phi((a_1,a_2,\dots,a_n))+\phi((b_1,b_2,\dots,b_n)).
\end{align*}
We see that $\phi$ preserves componentwise addition, so $\phi$ is an automorphism. Geometrically, $\phi$ represents a reflection over the origin. In $\mathbb{R}^2$ particularly, that is equivalent to a rotation by $\pi$ radians. \qed

\end{enumerate}

\end{document}