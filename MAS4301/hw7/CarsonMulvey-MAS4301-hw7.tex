\documentclass[11pt,letterpaper]{article}
%\usepackage{fullpage}
\usepackage{amsmath}
\usepackage{amsfonts}
\usepackage{amssymb}
\usepackage{amsthm}
\usepackage{array}
\usepackage[dvipsnames]{xcolor}
\usepackage{textcomp}

\linespread{1.15}

\newcommand{\bb}[1]{\mathbb{#1}}
\newcommand{\Mod}[1]{\ (\mathrm{mod}\ #1)}

\newenvironment{solution}{\color{blue}\textit{Solution.}}{\color{black}}

\begin{document}
\begin{center}
    \begin{large}
        \textbf{Homework 7} \\
        MAS4301 \\
        Carson Mulvey
    \end{large}
\end{center}

\begin{enumerate}
\item[\textbf{10.14}] Under $\varphi\colon \bb{Z}_{12} \to \bb{Z}_{10}$ mapping $x\mapsto 3x \Mod{10}$, we have
\[
    \varphi(4)+\varphi(8)=2+4=6,
\]
but
\[
    \varphi(4+8)=\varphi(0)=0.
\]
Since $\varphi(4+8)\neq \varphi(4)+\varphi(8)$, $\varphi$ does not preserve group operations, and therefore cannot be a homomorphism.
\item[\textbf{10.16}] Suppose such a homomorphism $\varphi$ from $\bb{Z}_8 \oplus \bb{Z}_2$ onto $\bb{Z}_4 \oplus \bb{Z}_4$ were to exist. Then since $|\bb{Z}_8 \oplus \bb{Z}_2|=|\bb{Z}_4 \oplus \bb{Z}_4|=16$, $\varphi$ has a domain and codomain of equal cardinality, so $\varphi$ must also be injective. Hence such a $\varphi$ would be an isomorphism. However, $\bb{Z}_8 \oplus \bb{Z}_2$ and $\bb{Z}_4 \oplus \bb{Z}_4$ are clearly not isomorphic, since $\bb{Z}_8 \oplus \bb{Z}_2$ has an element with order 8, (1,0), whereas $\bb{Z}_4 \oplus \bb{Z}_4$ has a maximal order of 4 for any element. Thus, we have a contradiction, so $\varphi$ cannot exist.
\item[\textbf{10.20}] 0 maps onto; 4 maps to
\item[\textbf{10.24}]
\begin{enumerate}
    \item $\varphi(x)=3x$
    \item $\text{im}(\varphi)=\{0,3,6,9,12\}$
    \item $\text{ker}(\varphi)=\{0,5,10,15,20,25,30,35,40,45\}$
    \item $\varphi^{-1}(3)=\{1,6,11,16,21,26,31,36,41,46\}$
\end{enumerate}
\end{enumerate}

\end{document}
