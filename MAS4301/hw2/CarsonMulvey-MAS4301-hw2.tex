\documentclass[11pt,letterpaper]{article}
%\usepackage{fullpage}
\usepackage{amsmath}
\usepackage{amsfonts}
\usepackage{amssymb}
\usepackage{amsthm}
\usepackage{array}
\usepackage[dvipsnames]{xcolor}
\usepackage{textcomp}

\newcommand{\N}{\mathbb{N}}

\newenvironment{solution}{\color{blue}\textit{Solution.}}{\color{black}}

\begin{document}
\begin{center}
    \begin{large}
        \textbf{Homework 2} \\
        MAS4301 \\
        Carson Mulvey
    \end{large}
\end{center}

\begin{enumerate}
\item[\textbf{NB 2.3}] We claim that all subgroups of $\mathbb{Z}$ take form $\langle a \rangle$ for some nonnegative integer $a$. In other terms, subgroups of $\mathbb{Z}$ are also cyclic. We will prove this by contradiction.

Suppose there exists a subgroup $H$ of $\mathbb{Z}$ that does not take the above form. Then for any choice of positive integer $a\in{H}$, $H \neq \langle a \rangle$ (the case of $a=0$ does not have to be checked, since $H\neq\{0\}$). Thus, for any choice of $a$, there exists $b\in{H}$ such that $b\notin{\langle a \rangle}$. This implies that $b\neq{0}$.

Then, since $H$ is a subgroup, $H$ is closed under the group operation, so for any $n,m\in{\mathbb{Z}}$, we know that $na+mb\in{H}$. We also know that for nonzero $a,b\in{\mathbb{Z}}$, there exist $n,m\in{\mathbb{Z}}$ such that $\text{gcd}(a,b)=na+mb$ Letting $n$ and $m$ be as such, we see that $\text{gcd}(a,b)\in{H}$. Again, since $H$ is closed under the group operation, $n\cdot\text{gcd}(a,b)\in{H}$ for all $n\in{\mathbb{Z}}$.

However, since all linear combinations of $a$ and $b$ can be expressed as $n\cdot\text{gcd}(a,b)$ for some $n\in{\mathbb{Z}}$, all elements of $H$ take this form. Thus $H=\langle \text{gcd}(a,b) \rangle$, which contradicts the assumption that $H$ does not take the form described above. Thus, all subgroups of $\mathbb{Z}$ must take the form described above. \qedsymbol

Other infinite cyclic groups are ``essentially the same'' as $\mathbb{Z}$, which lets us conclude that subgroups of infinite cyclic groups are also cyclic.

\item[\textbf{NB 2.4}] We will prove by contradiction. Suppose that $\mathbb{R}$ \textit{is} cyclic. Then there exists $a\in{\mathbb{R}}$ such that $\{na \mid n\in{\mathbb{Z}}\}=\mathbb{R}$. However, for any choice of $a$, we see that $\frac{1}{2}\cdot{a}$ is in $\mathbb{R}$ but \textit{not} in $\{na \mid n\in{\mathbb{Z}}\}$. This forms a contradiction, so we conclude that $\mathbb{R}$ is not cyclic. \qedsymbol

For all $a\in{Q}$, we also have $\frac{1}{2}\cdot{a}$ in $\mathbb{Q}$. Thus, the same argument made above can be made for $\mathbb{Q}$, replacing instances of $\mathbb{R}$ with $\mathbb{Q}$.
\item[\textbf{3.2}] In $Q$, we have $\langle{\frac{1}{2}}\rangle{} = \{\dots, -\frac{3}{2}, -1, -\frac{1}{2}, 0, \frac{1}{2}, 1, \frac{3}{2}, \dots\} = \{\frac{k}{2} \mid k\in{\mathbb{Z}}\}$.

In $Q^\ast$, we have $\langle{\frac{1}{2}}\rangle{} = \{\dots, 8, 4, 2, 1, \frac{1}{2}, \frac{1}{4}, \frac{1}{8}, \dots\} = \{\frac{1}{2^k} \mid k\in{\mathbb{Z}}\}$.
\item[\textbf{3.10}] There are $2$ subgroups of $D_4$: $\{R_0,R_{180},H,V\}$ and $\{R_0,R_{90},R_{180},R_{270}\}$.
\item[\textbf{3.18}] Let $G$ be a group, and let element $a\in{G}$ satisfy $a^6=e$. By the definition of order, $|a|\leq{6}$. Particularly, $a^6=e$ holds iff $|a|$ divides $6$, so we have $|a|\in\{1,2,3,6\}$.
\item[\textbf{3.28}] Let $G$ be a group containing elements $a$ and $b$ such that $|a|=|b|=2$ and $ab=ba$. We claim that $H=\{e,a,b,ab\}$ is a subgroup of $G$ of order $4$.

\noindent\begin{tabular}{c | c c c c}
    & e & a & b & ab
   \cr{}
   e & e & a & b & ab \\
   a & a & e & ab & b \\
   b & b & ab & e & a \\
   ab & ab & b & a & e \\
\end{tabular}

Using the Cayley table above, we see that $H$ is closed under the group operation. Additionally, Since $a^2=b^2=e$, we have $a^{-1}=a$ and $b^{-1}=b$. Along with $e^{-1}=e$ and $(ab)^{-1}=b^{-1}a^{-1}=ba=ab$, we see that $H$ is closed under taking inverses. Thus, $H$ is a subgroup of $G$, and has order $4$. \qedsymbol

\item[\textbf{3.32}] Since $\text{gcd}(2^{50},3^{50})=1$, there exists $n,m\in{\mathbb{Z}}$ such that $2^{50}\cdot{n}+3^{50}\cdot{m}=1$. Let $n,m$ be as such. Since $H$ is closed under the group operation of addition, $2^{50}\cdot{n}+3^{50}\cdot{m}=1\in{H}$. Since $1\in{H}$, which is a generator for $\mathbb{Z}$, it follows that $H=\mathbb{Z}$.
\item[\textbf{3.34}] Let there be a group $G$ with subgroups $H$ and $K$. We claim that $H\cap{K}$ is a subgroup of $G$. We will follow the definition of a subgroup as given in class (Two-Step Subgroup Test in the textbook).

Consider arbitrary elements $a,b\in{H\cap{K}}$. We see that $a,b\in{H}$, and because $H$ is a subgroup of $G$, it follows that $ab\in{H}$. Similarly, since $a,b\in{K}$ and $K$ is a subgroup of $G$, we have that $ab\in{K}$. Thus $ab\in{H\cap{K}}$, so $H\cap{K}$ is closed under the group operation.

Now consider an arbitrary element $a\in{H\cap{K}}$. Since $a\in{H}$ and $H$ is a subgroup of $G$, $a^{-1}\in{H}$. Similarly, since $a\in{K}$ and $K$ is a subgroup of $G$, $a^{-1}\in{K}$. Thus $a^{-1}\in{H\cap{K}}$, so $H\cap{K}$ is closed under taking inverses.

Because $H\cap{K}$ is closed under the group operation and inverses, we conclude that $H\cap{K}$ is a subgroup of $G$ by the definition of a subgroup. \qedsymbol
\end{enumerate}

\end{document}