\documentclass[11pt,letterpaper]{article}
%\usepackage{fullpage}
\usepackage{amsmath}
\usepackage{amsfonts}
\usepackage{amssymb}
\usepackage{amsthm}
\usepackage{array}
\usepackage[dvipsnames]{xcolor}
\usepackage{textcomp}

\linespread{1.15}

\newcommand{\N}{\mathbb{N}}
\newcommand{\Mod}[1]{\ (\mathrm{mod}\ #1)}

\newenvironment{solution}{\color{blue}\textit{Solution.}}{\color{black}}

\begin{document}
\begin{center}
    \begin{large}
        \textbf{Homework 6} \\
        MAS4301 \\
        Carson Mulvey
    \end{large}
\end{center}

\begin{enumerate}
\item[\textbf{NB 6.1}]
\begin{enumerate}
    \item[(c)] For $\mathbb{Z}_4\oplus \mathbb{Z}_2$, there is only one element of order 2, namely $(2,0)$. However, $D$, $D'$, $H$, $V$, and $R_{180}$ are all elements of order 2 in $D_4$. The number of elements of any particular order is preserved by all isomorphisms, so $D_4$ \textit{cannot} be isomorphic to $\mathbb{Z}_4\oplus \mathbb{Z}_2$.
    \item[(d)] In the case of $p=2$, we know that $D_4$ has order $p^3=8$, but $D_4$ is clearly not abelian (for instance, $R_{90}H\neq HR_{90}$).
\end{enumerate}
\item[\textbf{9.6}] Let $A=
\big(\begin{smallmatrix}
    0 & 1\\
    1 & 0
\end{smallmatrix}\big)\in GL(2,\mathbb{R})$ and $B=
\big(\begin{smallmatrix}
    a & b\\
    0 & d
\end{smallmatrix}\big)$
, with $a,b,d\in \mathbb{R}$ and $ad\neq 0$, be an element of $H$. By the 2x2 matrix inversion formula, we have $A^{-1}= \frac{1}{0-1}
\big(\begin{smallmatrix}
    0 & -1\\
    -1 & 0
\end{smallmatrix}\big) =
\big(\begin{smallmatrix}
   0 & 1\\
   1 & 0
\end{smallmatrix}\big)$. Then
\begin{align*}
    ABA^{-1}
    &=
    \begin{bmatrix}
        0 & 1\\
        1 & 0
    \end{bmatrix}
    \begin{bmatrix}
        a & b\\
        0 & d
    \end{bmatrix}
    \begin{bmatrix}
        0 & 1\\
        1 & 0
    \end{bmatrix} \\
    &=
    \begin{bmatrix}
        0 & d\\
        a & b
    \end{bmatrix}
    \begin{bmatrix}
        0 & 1\\
        1 & 0
    \end{bmatrix} \\
    &=
    \begin{bmatrix}
        d & 0\\
        b & a
    \end{bmatrix},
\end{align*}
but when $b\neq 0$, we see that $ABA^{-1}\notin H$. Thus it is \textit{false} that $AHA^{-1}\subseteq H$ for all $A\in GL(2,\mathbb{R})$, so $H$ is \textit{not} a normal subgroup of $A\in GL(2,\mathbb{R})$.
\item[\textbf{9.12}] Let $G$ be an abelian group, and $H$ be a normal subgroup of $G$. Let $x$ and $y$ be elements of $G/H$. We know that $x=aH$ and $y=bH$ for some $a,b\in G$. Then
\begin{align*}
    xy &= (aH)(bH) \\
    &= abH \\
    &= baH & \mbox{(since $G$ is abelian)}\\
    &= (bH)(aH) \\
    &= yx,
\end{align*}
so $G/H$ is clearly also abelian.
\item[\textbf{9.14}] $14+\langle 8 \rangle$ has order $4$.
\item[\textbf{9.18}] $\mathbb{Z}_{60}/\langle 15 \rangle$ has order $15$.
\item[\textbf{9.34}] Since 5 and 7 are relatively prime, we know that there exists integers $s$ and $t$ such that $5s+7t=1$. Let $s$ and $t$ be as such. Then for any $n\in \mathbb{Z}$, we have
\begin{align*}
    n\cdot 1&=n(5s+7t) \\
    &=5ns+7nt.
\end{align*}
Moreover, $5ns\in H$ and $7nt\in K$, so $n=5ns+7nt\in HK$. Thus $\mathbb{Z}\subseteq HK$. Also, $HK\subseteq \mathbb{Z}$ is clear, since an element in $HK$ is the sum of two integers. Thus $\mathbb{Z}=HK$. 

Also, since $\mathbb{Z}$ is abelian, $H$ and $K$ are normal subgroups. Additionally, $H\cap K=\{0\}$. Thus, by definition, $Z=H\times K$.
\end{enumerate}

\end{document}
