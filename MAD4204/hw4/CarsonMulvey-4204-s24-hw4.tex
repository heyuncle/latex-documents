\documentclass[11pt,letterpaper]{article}
\usepackage{fullpage}
\usepackage{multicol}
\usepackage{amsmath}
\usepackage{amsfonts}
\usepackage{amssymb}
\usepackage{amsthm}
\usepackage{graphicx, nicefrac}
\usepackage{tikz, nicefrac}

\newenvironment{solution}{\color{blue}\textit{Solution.}}{\color{black}}

\newcommand{\ds}{\displaystyle}
\newcommand{\bv}{\mathbf}
\newcommand{\lv}{\langle}
\newcommand{\rv}{\rangle}

\begin{document}

%%%Version 1%%%%

\flushleft

\begin{center}
    \begin{large}
        \textbf{Homework 4} \\
        MAD4204 \\
        Carson Mulvey
    \end{large}
\end{center}

\pagestyle{empty}


\flushleft

%Filling out the {\em online student data form} is worth an extra 4 points on this quiz.

\begin{enumerate}



\item For $P$ a finite poset, let $J(P)$ be the set of ideals in $P$ and $A(P)$ be the set of antichains.
\begin{enumerate}
	\item Find $\#J(P)$ and $\#A(P)$ for a chain. For an antichain.
	\item Find $\#J(P)$ and $\#A(P)$ for $B_3$.
	\item Must $\#J(P) = \#A(P)$? Why or why not? Explain.
\end{enumerate}
\begin{solution}
	\begin{enumerate}
		\item For chain $P$, all pairs of elements are comparable, so only singleton antichains exist (and the empty antichain). Thus $\#A(P)=\#P+1$.
		
		For an antichain $P$, all pairs of elements are incomparable, so any subset of $P$ is another antichain. Thus $\#A(P)=2^{\#P}$.
		\item test
	\end{enumerate}
\end{solution}


\item \begin{enumerate}
\item For $P$ a poset with $n$ elements, prove $P$ contains a chain with at least $\sqrt{n}$ elements or an antichain with at least $\sqrt{n}$ elements.
\item Prove Hall's theorem using Dilworth's theorem.
\end{enumerate}

\item For $P$ a finite poset, show the number of elements in a maximum chain equals the number of antichains in the smallest antichain cover.
 
\item Let $M(n,k)$ be the multiset consisting of $k$ copies of each element in $[n]$.
Let $P(n,k)$ be the poset on submultisets of $M(n,k)$ ordered by containment, e.g. 
\[
\{\{1,1,4\}\} \subseteq \{\{1,1,1,3,3,4,5,5\}\} \quad \mbox{but} \quad \{\{1,1,4\}\} \not\subseteq \{\{1,3,3,4,4\}\}.
\]
Find a general formula for $\mu_{P(n,k)}(x,y)$, and explain how it relates to Example 16.20.



\end{enumerate}

\end{document}