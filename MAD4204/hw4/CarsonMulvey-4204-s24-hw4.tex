\documentclass[11pt,letterpaper]{article}
\usepackage{fullpage}
\usepackage{multicol}
\usepackage{amsmath}
\usepackage{amsfonts}
\usepackage{amssymb}
\usepackage{amsthm}
\usepackage{graphicx, nicefrac}
\usepackage{tikz, nicefrac}

\newenvironment{solution}{\color{blue}\textit{Solution.}}{\color{black}}

\newcommand{\ds}{\displaystyle}
\newcommand{\bv}{\mathbf}
\newcommand{\lv}{\langle}
\newcommand{\rv}{\rangle}

\begin{document}

%%%Version 1%%%%

\flushleft

\begin{center}
    \begin{large}
        \textbf{Homework 4} \\
        MAD4204 \\
        Carson Mulvey
    \end{large}
\end{center}

\pagestyle{empty}


\flushleft

%Filling out the {\em online student data form} is worth an extra 4 points on this quiz.

\begin{enumerate}



\item For $P$ a finite poset, let $J(P)$ be the set of ideals in $P$ and $A(P)$ be the set of antichains.
\begin{enumerate}
	\item Find $\#J(P)$ and $\#A(P)$ for a chain. For an antichain.
	\item Find $\#J(P)$ and $\#A(P)$ for $B_3$.
	\item Must $\#J(P) = \#A(P)$? Why or why not? Explain.
\end{enumerate}

\begin{solution}
	\begin{enumerate}
		\item For chain $P$, each element generates a unique ideal. Conversely, all ideals in $P$ can be traced to a unique maximum element. Thus, including the empty ideal, $\#J(P)=\#P+1$. Also, all pairs of elements are comparable, so only singleton antichains exist (plus the empty antichain). Thus $\#A(P)=\#P+1$.
		
		For an antichain $P$, all pairs of elements are incomparable, so any subset of $P$ is another antichain. Since no element is strictly less than another, all subsets of $P$ are also ideals. Conversely, ideals and antichains of $P$ must be subsets of $P$. Thus $\#J(P)=\#A(P)=2^{\#P}$.

		\item For $P=B_3$, we look at antichains $\{\{1\},\{2\},\{3\}\}$ and $\{\{1,2\},\{2,3\},\{3,1\}\}$, each of which have $8$ antichain subsets. Since the empty set is counted twice, this gives $15$ antichains. Besides this, we have $\{\varnothing\}$, $\{\{1,2,3\}\}$, $\{\{1\},\{2,3\}\}$, $\{\{2\},\{3,1\}\}$, and $\{\{3\},\{1,2\}\}$, for a total of $\#A(P)=20$. All ideals come from extending these antichains to include all subsets of its elements, so $\#J(P)=20$.
		\item Yes, $\#J(P) = \#A(P)$ must hold! We will describe a process that creates a bijection between ideals and antichains.

		For any ideal $I\subseteq{P}$, denote $\tilde{I}$ as the set of maximal elements of $I$. We note that for any pair $x,y\in{\tilde{I}}$, $x$ and $y$ are incomparable, since either $x>y$ or $y>x$ would make one of $x$ and $y$ not maximal. Thus $\tilde{I}$ is an antichain.

		Conversely, let $A$ be an antichain. Define $\tilde{A}$ to be the set where $x\in{\tilde{A}}$ if $x\leq{a}$ for some $a\in{A}$. If $y\leq{x}$, then by transitivity, $y\leq{a}$ for some $a\in{A}$, so $y\in{\tilde{A}}$. This makes $\tilde{A}$ an ideal by definition.
	\end{enumerate}
\end{solution}


\item \begin{enumerate}
\item For $P$ a poset with $n$ elements, prove $P$ contains a chain with at least $\sqrt{n}$ elements or an antichain with at least $\sqrt{n}$ elements.
\item Prove Hall's theorem using Dilworth's theorem.
\end{enumerate}

\begin{solution}
	\begin{enumerate}
		\item Let poset $P$ have no antichain with at least $\sqrt{n}$ elements. Then let the width of $P$ be $a<\sqrt{n}$. By Dilworth's Theorem, the number of elements in a minimal chain cover is also $a$. Then by the Pigeonhole Principle, at least one chain in any chain cover must contain at least $\lceil n/a \rceil$ elements. But since $a<\sqrt{n}$, we have 
		\begin{align*}
			\lceil n/a \rceil &\geq \lceil \sqrt{n} \rceil \\
			&\geq \sqrt{n}.
		\end{align*}
		Thus $P$ contains a chain with at least $\sqrt{n}$ elements. \qed
	\end{enumerate}
\end{solution}

% \item For $P$ a finite poset, show the number of elements in a maximum chain equals the number of antichains in the smallest antichain cover.

% \begin{solution}
% 	Let $a$ be the size of a minimal antichain cover of $P$ and $b$ be the size of a maximal chain in $P$. 
% \end{solution}
 
\item[4.] Let $M(n,k)$ be the multiset consisting of $k$ copies of each element in $[n]$.
Let $P(n,k)$ be the poset on submultisets of $M(n,k)$ ordered by containment, e.g. 
\[
\{\{1,1,4\}\} \subseteq \{\{1,1,1,3,3,4,5,5\}\} \quad \mbox{but} \quad \{\{1,1,4\}\} \not\subseteq \{\{1,3,3,4,4\}\}.
\]
Find a general formula for $\mu_{P(n,k)}(x,y)$, and explain how it relates to Example 16.20.

\begin{solution}
	We know that $a|b$ iff for any prime $p$, the exponent of $p$ in the prime factorization in $a$ is less than or equal to that of $b$. Thus, taking the number of copies of some $i$ in a multiset as the power of prime $p_i$ in an integer, we have a mapping between $P(n,k)$ and $(\mathbb{N},|)$, where $n,k$ are arbitrarily large as needed. Thus, $\mu_{P(n,k)}(x,y)=(-1)^n$ if there is exactly 1 more copy of each element in $y$ than that in $x$, and $\mu_{P(n,k)}(x,y)=0$ otherwise.
\end{solution}


\end{enumerate}

\end{document}