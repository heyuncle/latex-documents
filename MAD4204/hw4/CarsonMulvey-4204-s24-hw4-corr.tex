\documentclass[11pt,letterpaper]{article}
\usepackage{fullpage}
\usepackage{multicol}
\usepackage{amsmath}
\usepackage{amsfonts}
\usepackage{amssymb}
\usepackage{amsthm}
\usepackage{graphicx, nicefrac}
\usepackage{tikz, nicefrac}

\newenvironment{solution}{\color{blue}\textit{Solution.}}{\color{black}}

\newcommand{\ds}{\displaystyle}
\newcommand{\bv}{\mathbf}
\newcommand{\lv}{\langle}
\newcommand{\rv}{\rangle}

\begin{document}

%%%Version 1%%%%

\flushleft

\begin{center}
    \begin{large}
        \textbf{Homework 4 Revisions} \\
        MAD4204 \\
        Carson Mulvey
    \end{large}
\end{center}

\pagestyle{empty}


\flushleft

%Filling out the {\em online student data form} is worth an extra 4 points on this quiz.

\begin{enumerate}
	\item[3.] For $P$ a finite poset, show the number of elements in a maximum chain equals the number of antichains in the smallest antichain cover.

	\begin{solution}
		Let the \textit{height} of $y\in P$ be the maximum length of a saturated chain
		\[
			x\lessdot p_1 \lessdot p_2 \lessdot \cdots \lessdot y,
		\]
		with $x$ a minimal element of $P$. Then let $A_i$ be the set of elements of height $i$. We claim that for any $1\leq i\leq h$, $A_i$ is an antichain. Indeed, note that if $x,y\in A_i$ have $x<y$, then we can extend the chain of length $i$ ending at $x$ instead to end at $y$. This means that the height of $y$ must be greater than $i$, so $y\notin{A_i}$, forming a contradiction.
		
		Now let $C$ be a maximal chain of size $h$. Then since $h$ is also the largest height of any element in $P$, we see that $\bigcup_{i=1}^h{A_h}$ is an antichain cover of size $n$. Additionally, if an antichain cover of size $g<h$ were to exist, then some antichain must contain at least $2$ distict elements in $C$, since $|C|=h$. However, these two elements would be comparable, forming a contradiction. Thus, the smallest antichain cover has size $h$.
	\end{solution}
\end{enumerate}

\end{document}