\documentclass[11pt,letterpaper]{article}
\usepackage{fullpage}
\usepackage{amsmath}
\usepackage{amsfonts}
\usepackage{amssymb}
\usepackage[dvipsnames]{xcolor}

\newenvironment{solution}{\color{blue}\textit{Solution.}}{\color{black}}

\begin{document}
\begin{center}
    \begin{large}
        \textbf{Homework 1} \\
        MAD4204 \\
        Carson Mulvey
    \end{large}
\end{center}

\begin{enumerate}

    \item Let $G = ([n],E)$ be a graph and let $\overline{G} = ([n], \binom{[n]}{2} \setminus E)$ be its complement.
    Prove for $n$ sufficiently large that at least one of $G$ and $\overline{G}$ contains a cycle.
    
    Your proof should include a value $n$ that guarantees this property.

    \begin{solution}
        Let $n=5$. We first show that if neither $G$ nor $\overline{G}$ contain a cycle, then $G$ must be connected. 
        
        
        Suppose $G$ is disconnected. In one case, $G$ contains a vertex with no neighbors. Without loss of generality, let $|N(1)|=0$. If the other four vertices form a complete subgraph, then a cycle trivially exists. Hence, there must exist two vertices $a$ and $b$ where $(a,b)\notin{E}$. Then $\{1,a,b\}$ forms a cycle in $\overline{G}$. 
        
        The other possible disconnected graph has two connected subgraphs, one connecting $2$ vertices, the other connecting $3$ vertices. Since the subgraph of cardinality $3$ cannot contain a cycle, there must be two disjoint vertices, say $a$ and $b$. Then let an arbitrary vertex in the subgraph of cardinality $2$ be $c$. Since the two subgraphs are not connected, $\{a,b,c\}$ must form a cycle in $\overline{G}$. Thus $G$ is connected. If $G$ contains no cycles and is connected, then it must be a tree.

        Let $S=\{s\in{[n]}:N(s)=1\}$. If $|S|=2$, then $G$ is isomorphic to $H=([5],\{(1,2),(2,3),(3,4),(4,5)\})$. The equivalent to $\{1,3,5\}$ would form a cycle in $\overline{G}$. Otherwise, $|S|\geq3$. Since $|N(s)|=1$ for all $s\in{S}$, an edge connecting two elements of $S$ would make the graph disconnected. Thus, no two elements of $S$ can be connected by an edge. Thus $3$ arbitrary elements in $S$ form a cycle in $\overline{G}$. Thus for $n=5$, at least one of $G$ and $\overline{G}$ must contain a cycle.

        For any graph with $n>5$, any induced subgraph of $5$ vertices must follow the same property. Thus for all $n\geq{5}$, the property holds.


    \end{solution}
    
    \item Let $G = ([n],E)$ be a finite simple graph.
    Let $M$ be a maximal matching in $G$ and $M'$ be a maximum matching in $G$.
    Prove that $|M'| \leq 2 |M|$.
    
    
    \item For $\mathcal{M}$ a matroid on ground set $S$ defined in terms of bases, we say $I \subseteq S$ is \emph{independent} if there exists $B \in \mathcal{M}$ so that $I \subseteq B$.
    An alternate definition of a matroid $\mathcal{M}_I$ on ground set $S$ in terms of independent sets is that $\mathcal{M}_I \subseteq 2^{S}$ so that:
    \begin{itemize}
        \item (hereditary property) if $A \subseteq B \in \mathcal{M}_I$, then $A \in \mathcal{M}_I$,
        \item (augmentation property) for $A, B \in \mathcal{M}_I$ with $|A| < |B|$ there exists $b \in B$ so that $A \cup \{b\} \in \mathcal{M}_I$.
    \end{itemize}
    We show these definitions are equivalent by solving:
    \begin{enumerate}	
        \item For $\mathcal{M}_I$ a matroid defined in terms of its independent sets, we say $B \in \mathcal{M}_I$ is a \emph{basis} if $B$ is maximal in $\mathcal{M}_I$.
        Prove two bases in $\mathcal{M}_I$ satisfy the exchange property.

        \begin{solution}
            We first show that elements of $\mathcal{M}_I$ are bases iff they have the same cardinality as other bases. Let there be two bases $A,B\in\mathcal{M}_I$ such that, without loss of generality, $|A|<|B|$. Then by the augmentation property, there exists $b \in B\setminus{A}$ such that $A \cup \{b\} \in \mathcal{M}_I$. However, since $A$ is maximal, $A\subseteq{A\cup\{b\}}$ implies $A=A\cup\{b\}$, which can't be true as $b\notin{A}$. Therefore, all bases in $\mathcal{M}_I$ must have the same cardinality. 
            
            For the other direction, let there be basis $A\in\mathcal{M}_I$ and non-basis $B\in\mathcal{M}_I$ so that $|A|=|B|$. Then since $B$ is not maximal, there exists a set $C$ such that $B\cup{C}\in\mathcal{M}_I$ would be a basis. However, $|A|<|B\cup{C}|$ contradicts all bases having the same cardinality. Thus, an independent set with the same cardinality of a basis is another basis.

            Now let $A,B\in\mathcal{M}_I$ be bases, and $a\in{A}$. By the hereditary property, $A\setminus\{a\}\in{\mathcal{M}_I}$. Then, since all bases have the same cardinality, $|A\setminus\{a\}|=|B|-1$, so $|A|<|B|$. Thus, by the augmentation property, there exists $b \in B$ so that $(A\setminus\{a\}) \cup \{b\} \in \mathcal{M}_I$. Since one element was removed and a different element was added to $A$, we have $|A|=|(A\setminus\{a\}) \cup \{b\}|$. Thus $(A\setminus\{a\}) \cup \{b\}$ is a basis, and so the exchange property holds.
        \end{solution}
        
        \item For $\mathcal{M}_B$ a matroid defined in terms of its bases, show that two independent sets $A,B$ of $\mathcal{M}$ satisfy the augmentation property.
        
        (Compare this with Lemma 10.10 in the course text)
        \begin{solution}
            Let $A,B$ be two independent sets of $\mathcal{M}$ with $|A|<|B|$. Then let $X_A,X_B\in\mathcal{M}_B$ such that $A\subseteq X_A$ and $B\subseteq X_B$. Then by the exchange property, for some $a\in{A}$, we have $(X_A\setminus\{a\})\cup \{b\}$ for some $b\in{X_B}$. Assuming the hereditary property, since $A\cup\{b\}\subseteq (X_A\setminus\{a\})\cup \{b\}$, we have $A\cup\{b\}$ an independent set, so the augmentation property holds.
        \end{solution}

    \end{enumerate}
     
    \item For $\mathcal{M}$ a matroid on ground set $S$ (from Problem 3, we can define it in terms of bases or independent sets, whichever is more convenient), let $w:S \to \mathbb{R}_{\geq 0}$.
    For $B$ a basis in $\mathcal{M}$, define $w(B) = \sum_{b \in B} w(b)$.
    
    Describe an algorithm for finding the basis $B$ minimizing $w(B)$ and prove that it is optimal.
    
    (Hint: compare to the greedy algorithm for finding the minimum weighted spanning tree)

\end{enumerate}    

\end{document}