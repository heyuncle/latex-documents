\documentclass[11pt,letterpaper]{article}
\usepackage{fullpage}
\usepackage{amsmath}
\usepackage{amsfonts}
\usepackage{amssymb}
\usepackage{amsthm}
\usepackage{array}
\usepackage[dvipsnames]{xcolor}
\usepackage{textcomp}

\newcommand{\N}{\mathbb{N}}
\newcommand{\R}{\mathbb{R}}

\newenvironment{solution}{\color{blue}\textit{Solution.}}{\color{black}}

\begin{document}
\begin{center}
    \begin{large}
        \textbf{Assignment 4} \\
        MAA4211 \\
        Carson Mulvey
    \end{large}
\end{center}

\begin{enumerate}
    \item[\textbf{(Graded) 2.2.6.}] Let $(a_n)$ be a sequence that converges to $a$ and converges to $b$. Now, let $\epsilon>0$. There must exist an $N_1$ such that for all $n\geq N_1$, we have $|a_n-a| < \epsilon/2$. There must also exist an $N_2$ such that for all $n \geq N_2$, we have $|a_n-b| < \epsilon/2$. Then, for all $n \geq \max\{N_1,N_2\}$,
    \begin{align*}
        |a-b| &= |a-b+a_n-a_n| \\
        &\leq |a_n-a| + |a_n-b| \qquad\text{(triangle inequality)}
        \\ 
        &< \epsilon/2 + \epsilon/2 = \epsilon.
    \end{align*}
    However, this implies that $|a-b|<\epsilon$ for all $\epsilon>0$. By Theorem 1.2.6 from the book, this shows that $a=b$. \qed

    \item[\textbf{2.5.1.}]
\begin{enumerate}
    \item Let $(a_n)$ be a sequence with a bounded subsequence $(a_{n_1},a_{n_2},\dots)$. By the Bolzano--Weierstrass Theorem, $(a_{n_1},a_{n_2},\dots)$ must itself have a convergent subsequence. However, this convergent subsequence is transitively also a subsequence of $(a_n)$, hence making (a) impossible.

    \item Let $(a_n)$ be a sequence such that
    \[
        a_k = (-1)^{k}\left(\frac{k-1}{2k}\right)+\frac{1}{2}.
    \]
    Then the subsequence $(a_1,a_3,a_5,\dots)$ converges to 0, while $(a_2,a_4,a_6,\dots)$ converges to 1. Additionally, all $a_k$ are in the open interval $(0,1)$, so this sequence satisfies (b).

    \item Let $(a_n)$ be a sequence such that 
    \[
        a_k = \prod_{i=1}^\infty \frac{p_{i}^{m_{i}-1}-1}{p_{i}^{m_{i}}},
    \]
    where $p_1^{m_1}p_2^{m_2}\cdots$ is the unique prime factorization of $k$. We note that $(p^{i(m-1)}-1)/{p^{im}}$ as a sequence indexed by $i$ converges to $1/p^m$.

    \item Let $(a_n)$ have subsequences converging to each element of $\{1/i : i\in\N\}$. We claim that such an $(a_n)$ must converge to 0.
    Note: I could not find justification. Below is my attempted proof.

    \color{lightgray}Let $\epsilon>\epsilon' > 0$. For a subsequence converging to some $1/i$, we know that there exists an $N_i$ such that all natural numbers $n\geq N_i$ \textit{whose indices are in the subsequence} must satisfy $|a_n-1/i|<\epsilon'=\epsilon+1/k$. We now pick large enough $k$ so that $\max\{N_1,N_2,\dots,N_k\}$... :(\color{black}
\end{enumerate}
\item[] 


\end{enumerate}

\end{document}