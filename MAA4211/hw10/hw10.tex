\documentclass[11pt,letterpaper]{article}
\usepackage{fullpage}
\usepackage{amsmath}
\usepackage{amsfonts}
\usepackage{amssymb}
\usepackage{amsthm}
\usepackage{array}
\usepackage[dvipsnames]{xcolor}
\usepackage{textcomp}

\newcommand{\N}{\mathbb{N}}
\newcommand{\R}{\mathbb{R}}

\newenvironment{solution}{\color{blue}\textit{Solution.}}{\color{black}}

\linespread{1.15}

\begin{document}
\begin{center}
    \begin{large}
        \textbf{Assignment 10} \\
        MAA4211 \\
        Carson Mulvey
    \end{large}
\end{center}

\begin{enumerate}
    \item[\textbf{(Graded) 5.3.3.}]
        \begin{enumerate}
            \item Let $g(x) = h(x) - x$. Because $h(x)$ is differentiable on $[0,3]$, it is also continuous on $[0,3]$, making $g$ also continuous on $[0,3]$ by the Algebraic Continuity Theorem.
            
            We can compute $g(0) = h(0) - 0 = 1$ and $g(3) = h(3) - 3 = -1$. In other terms, $g(3) < 0 < g(0)$, so by the Intermediate Value Theorem, there must exist $d\in (0,3) \subset [0,3]$ such that $g(d)=0$, or equivalently, $h(d)=d$.

            \item We know that $h$ is differentiable (and hence continuous) on $[0,3]$. Then, by the Mean Value Theorem, there exists a point $c\in (0,3)$ where
            \[ h'(c) = \frac{h(3)-h(0)}{3-0} = \frac{1}{3}. \]
            
            \item Let $g(x) = h(x) - \frac{1}{4}x$. Similar to in (a), we can deduce that $g$ is continuous and differentiable on $[0,3]$. Additionally, we have $g(0)=3/4$, $g(1)=7/4$, and $g(3)=5/4$.
            
            But since $g(0) < 5/4 < g(1)$, by the Intermediate Value Theorem, we must have some $c\in (0,1)$ such that $g(c)=5/4$. Then, since $g(c)=g(3)$, by Rolle's Theorem, we must have some $d\in (c, 3) \subset [0,3]$ where $g'(d)=0$. Because $g'(x)=h'(x)-\frac{1}{4}$, this is equivalent to $h'(d)=1/4$, as desired.
        \end{enumerate}
\end{enumerate}


\end{document}