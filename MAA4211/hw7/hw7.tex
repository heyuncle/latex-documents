\documentclass[11pt,letterpaper]{article}
\usepackage{fullpage}
\usepackage{amsmath}
\usepackage{amsfonts}
\usepackage{amssymb}
\usepackage{amsthm}
\usepackage{array}
\usepackage[dvipsnames]{xcolor}
\usepackage{textcomp}

\newcommand{\N}{\mathbb{N}}
\newcommand{\R}{\mathbb{R}}

\newenvironment{solution}{\color{blue}\textit{Solution.}}{\color{black}}

\begin{document}
\begin{center}
    \begin{large}
        \textbf{Assignment 7} \\
        MAA4211 \\
        Carson Mulvey
    \end{large}
\end{center}

\begin{enumerate}
    \item[\textbf{(Graded) 3.3.3.}]
        Let $K \subseteq \R$ be closed and bounded. Consider a sequence $(a_n)$ in $K$. Because $K$ is bounded by some $M$, all elements $a_k\leq M$, so $(a_n)$ is also bounded. Thus, by Bolzano-Weierstrass, $(a_n)$ has a subsequence that converges, say $(a_{n_k})$.

        However, this convergent subsequence must also be a Cauchy sequence. Because $K$ is closed, by Theorem 3.2.8., $(a_{n_k})$ has a limit that is also in $K$. Thus, by the definition of compactness, $K$ is compact.
\end{enumerate}


\end{document}