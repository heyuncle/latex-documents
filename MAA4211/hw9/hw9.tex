\documentclass[11pt,letterpaper]{article}
\usepackage{fullpage}
\usepackage{amsmath}
\usepackage{amsfonts}
\usepackage{amssymb}
\usepackage{amsthm}
\usepackage{array}
\usepackage[dvipsnames]{xcolor}
\usepackage{textcomp}

\newcommand{\N}{\mathbb{N}}
\newcommand{\R}{\mathbb{R}}

\newenvironment{solution}{\color{blue}\textit{Solution.}}{\color{black}}

\begin{document}
\begin{center}
    \begin{large}
        \textbf{Assignment 9} \\
        MAA4211 \\
        Carson Mulvey
    \end{large}
\end{center}

\begin{enumerate}
    \item[\textbf{(Graded) 4.5.7.}]
        Consider the function $g(x)=f(x)-x$. By the Algebraic Continuity Theorem, $g$ is continuous on interval $[0,1]$. A fixed point of $f$ will occur for some $x\in [0,1]$ iff $g(x)=0$.
        
        Using the range of $f$, we know that $0\leq g(0)\leq 1$ and $-1\leq g(1)\leq 0$. If $g(0)=0$ or $g(1)=0$, then a fixed point occurs at $x=0$ or $x=1$, respectively. Now let $g(0)>0$ and $g(1)<0$. We have $g(0)>0>g(1)$, so by the Intermediate Value Theorem, there is a point $c\in (0,1)$ where $g(c)=0$, thus making $x=c$ a fixed point.
\end{enumerate}


\end{document}