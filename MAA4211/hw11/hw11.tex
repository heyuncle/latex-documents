\documentclass[11pt,letterpaper]{article}
\usepackage{fullpage}
\usepackage{amsmath}
\usepackage{amsfonts}
\usepackage{amssymb}
\usepackage{amsthm}
\usepackage{array}
\usepackage[dvipsnames]{xcolor}
\usepackage{textcomp}

\newcommand{\N}{\mathbb{N}}
\newcommand{\R}{\mathbb{R}}

\newenvironment{solution}{\color{blue}\textit{Solution.}}{\color{black}}

\linespread{1.15}

\begin{document}
\begin{center}
    \begin{large}
        \textbf{Revisions for Problem 7.2.3} \\
        MAA4211 \\
        Carson Mulvey
    \end{large}
\end{center}

\begin{enumerate}
    \item[(a)] ($\Leftarrow$) Suppose there exists a sequence of partitions $(P_n)_{n=1}^\infty$ where
    \[ \lim_{n\rightarrow \infty} [ U(f,P_n) - L(f,P_n) ] = 0. \]
    Let $\epsilon >0$. By our limit, there must exist some sufficiently large $n$ such that
    \[
        U(f) - L(f) \leq U(f,P_n) - L(f,P_n) < \epsilon.
    \]
    Because $\epsilon$ is arbitrary, we have $U(f)=L(f)$, making $f$ integrable.

    ($\Rightarrow$) Now suppose that $f$ is integrable on $[a,b]$. Let $k$ be a positive integer. By the Integrability Criterion, there must exist a partition $P_k$ of $[a,b]$ such that
    \[ U(f,P_k) - L(f,P_k) < \frac{1}{k}. \]
    Consider the sequence $(P_n)_{n=1}^\infty$ formed by taking the partition described above for each positive integer. Let $\epsilon >0$. We can pick sufficiently large $n$ such that
    \[ U(f,P_n) - L(f,P_n) < \frac{1}{n} < \epsilon. \]
    Thus, for $(P_n)_{n=1}^\infty$, we have
    \[ \lim_{n\rightarrow \infty} [ U(f,P_n) - L(f,P_n) ] = 0, \]
    as desired.
    
    
    \item[(b)] Because $P_n$ divides $[0,1]$ into $n$ equal subintervals, we have $x_k = \frac{k}{n}$. Because $f(x)=x$, we have \[ m_k=\inf [x_{k-1},x_k] = x_{k-1} = \frac{k-1}{n} \]    
    and
    \[ M_k=\sup [x_{k-1},x_k] = x_k = \frac{k}{n}. \]
    This gives
    \begin{align*}
        U(f,P_n) &= \sum_{k=1}^n M_k(x_k-x_{k-1}) \\
        &= \sum_{k=1}^n \frac{k}{n} \left(\frac{k}{n}-\frac{k-1}{n} \right) \\
        &= \frac{1}{n^2} \sum_{k=1}^n k \\
        &= \frac{1}{n^2} \cdot \frac{n(n+1)}{2} \\
        &= \frac{n+1}{2n}.
    \end{align*}
    Similarly,
    \begin{align*}
        L(f,P_n) &= \sum_{k=1}^n m_k(x_k-x_{k-1}) \\
        &= \sum_{k=1}^n \frac{k-1}{n} \left(\frac{k}{n}-\frac{k-1}{n} \right) \\
        &= \frac{1}{n^2} \sum_{k=1}^n (k-1) \\
        &= \frac{1}{n^2} \cdot \frac{(n-1)\cdot n}{2} \\
        &= \frac{n-1}{2n}.
    \end{align*}
    \item[(c)] Using the formulas from (b), we have
    \begin{align*}
        \lim_{n\rightarrow \infty} [ U(f,P_n) - L(f,P_n) ]
        &= \lim_{n\rightarrow \infty} \left[ \frac{n+1}{2n} - \frac{n-1}{2n} \right] \\
        &= \lim_{n\rightarrow \infty} \frac{1}{n} \\
        &= 0.
    \end{align*}
    Hence, by part (a), we know that $f$ is integrable on $[0,1]$. Then,
    \begin{align*}
        \int_0^1 f &= \lim_{n\rightarrow\infty} U(f,P_n) \\
        &= \lim_{n\rightarrow\infty} \frac{n+1}{2n} \\
        &= \frac{1}{2}.
    \end{align*}
\end{enumerate}


\end{document}